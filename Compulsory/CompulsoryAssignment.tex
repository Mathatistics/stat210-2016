\documentclass[11pt,a4paper]{article}
\usepackage{lmodern}
\usepackage{amssymb,amsmath}
\usepackage{ifxetex,ifluatex}
\usepackage{fixltx2e} % provides \textsubscript
\ifnum 0\ifxetex 1\fi\ifluatex 1\fi=0 % if pdftex
  \usepackage[T1]{fontenc}
  \usepackage[utf8]{inputenc}
\else % if luatex or xelatex
  \ifxetex
    \usepackage{mathspec}
  \else
    \usepackage{fontspec}
  \fi
  \defaultfontfeatures{Ligatures=TeX,Scale=MatchLowercase}
  \newcommand{\euro}{€}
\fi
% use upquote if available, for straight quotes in verbatim environments
\IfFileExists{upquote.sty}{\usepackage{upquote}}{}
% use microtype if available
\IfFileExists{microtype.sty}{%
\usepackage{microtype}
\UseMicrotypeSet[protrusion]{basicmath} % disable protrusion for tt fonts
}{}
\usepackage[margin=1in]{geometry}
\usepackage{hyperref}
\PassOptionsToPackage{usenames,dvipsnames}{color} % color is loaded by hyperref
\hypersetup{unicode=true,
            pdftitle={Compulsory Assignment},
            pdfauthor={Raju Rimal},
            pdfsubject={Design and Analysis of Experiment (STAT210)},
            pdfborder={0 0 0},
            breaklinks=true}
\urlstyle{same}  % don't use monospace font for urls
\usepackage{longtable,booktabs}
\usepackage{graphicx,grffile}
\makeatletter
\def\maxwidth{\ifdim\Gin@nat@width>\linewidth\linewidth\else\Gin@nat@width\fi}
\def\maxheight{\ifdim\Gin@nat@height>\textheight\textheight\else\Gin@nat@height\fi}
\makeatother
% Scale images if necessary, so that they will not overflow the page
% margins by default, and it is still possible to overwrite the defaults
% using explicit options in \includegraphics[width, height, ...]{}
\setkeys{Gin}{width=\maxwidth,height=\maxheight,keepaspectratio}
\setlength{\parindent}{0pt}
\setlength{\parskip}{6pt plus 2pt minus 1pt}
\setlength{\emergencystretch}{3em}  % prevent overfull lines
\providecommand{\tightlist}{%
  \setlength{\itemsep}{0pt}\setlength{\parskip}{0pt}}
\setcounter{secnumdepth}{0}

%%% Use protect on footnotes to avoid problems with footnotes in titles
\let\rmarkdownfootnote\footnote%
\def\footnote{\protect\rmarkdownfootnote}

%%% Change title format to be more compact
\usepackage{titling}

% Create subtitle command for use in maketitle
\newcommand{\subtitle}[1]{
  \posttitle{
    \begin{center}\large#1\end{center}
    }
}

\setlength{\droptitle}{-2em}
  \title{Compulsory Assignment}
  \pretitle{\vspace{\droptitle}\centering\huge}
  \posttitle{\par}
\subtitle{Design and Analysis of Experiment (STAT210)}
  \author{Raju Rimal}
  \preauthor{\centering\large\emph}
  \postauthor{\par}
  \predate{\centering\large\emph}
  \postdate{\par}
  \date{19 August 2016}


% Redefines (sub)paragraphs to behave more like sections
\ifx\paragraph\undefined\else
\let\oldparagraph\paragraph
\renewcommand{\paragraph}[1]{\oldparagraph{#1}\mbox{}}
\fi
\ifx\subparagraph\undefined\else
\let\oldsubparagraph\subparagraph
\renewcommand{\subparagraph}[1]{\oldsubparagraph{#1}\mbox{}}
\fi

\usepackage{mathpazo}
\usepackage{setspace}
\usepackage{enumitem}
\usepackage[usenames,dvipsnames,svgnames,table]{xcolor}
\onehalfspacing
\setlength{\parindent}{0em}
\setlist[enumerate]{leftmargin=*}

\begin{document}
\maketitle

Show Answers

Hide Answers

NA

Please limit computer output as much as possible in your paper, and any
computer output should be commented.

\section{Exercise 1}\label{exercise-1}

You may use R or other statistical software for this exercise. The data
for this exercise is called \texttt{frystemp} and can be loaded from the
file \texttt{comp1.frystemp.RData} (all data sets are in the \emph{data}
folder of fronter). A lab has investigated the freezing temperature
(called \texttt{frystemp} in the dataset reproduced below) for three
different brands (called \texttt{merke} in the dataset).

\begin{enumerate}
\def\labelenumi{\alph{enumi})}
\item
  Calculate mean and standard deviation for each brand. Calculate the
  overall mean (mean for all observations).

  \begin{longtable}[c]{@{}lrrr@{}}
  \caption{Numerical Summary of \texttt{frystemp} data}\tabularnewline
  \toprule
  & Mean & SD & N\tabularnewline
  \midrule
  \endfirsthead
  \toprule
  & Mean & SD & N\tabularnewline
  \midrule
  \endhead
  merke1 & -16.163 & 1.309 & 8\tabularnewline
  merke2 & -13.562 & 1.283 & 8\tabularnewline
  merke3 & -14.575 & 0.755 & 8\tabularnewline
  Overall & -14.767 & 1.546 & 24\tabularnewline
  \bottomrule
  \end{longtable}
\item
  Consider the following model:

  \[y_{ij} = \mu + \tau_i + \epsilon_{ij}, \; \text{ where, } \sum_{i = 1}^3\tau_i = 0\]

  Here,

  \[
  \begin{aligned}
  y_{ij} &: \text{freezing temperature for each brand, } i = 1, 2, 3 \text{ and } j = 1, \ldots, 8 \\
  \tau_i &: \text{effect of brand }i
  \end{aligned}
  \]

  State the standard assumptions of the model. Use ANOVA to test if
  there is a difference between the brands. Formulate the hypothesis,
  carry out a test and formulate a conclusion. Give the ANOVA table.

  The standard assumption of the model is,
  \[\epsilon_{ij} \sim N(0, \sigma^2)\]

  It is also assumed that the \(y_{ij}\) are independent. The hypothesis
  for testing if there is a difference between the brands is,

  \[
  \begin{aligned}
  H_0 &: \tau_i = 0 \text{ for all } i \\
  H_1 &: \tau_i \ne 0 \text{ for at least one } i
  \end{aligned}
  \]

  The ANOVA table for the model is,

  \begin{longtable}[c]{@{}lrrrrr@{}}
  \caption{ANOVA table for \texttt{frystemp} model}\tabularnewline
  \toprule
  & Df & Sum Sq & Mean Sq & F value & Pr(\textgreater{}F)\tabularnewline
  \midrule
  \endfirsthead
  \toprule
  & Df & Sum Sq & Mean Sq & F value & Pr(\textgreater{}F)\tabularnewline
  \midrule
  \endhead
  merke & 2 & 27.48083 & 13.740417 & 10.48791 & 0.0006947\tabularnewline
  Residuals & 21 & 27.51250 & 1.310119 & &\tabularnewline
  \bottomrule
  \end{longtable}

  Here, p-value \(\left(0.001\right)\) is much smaller than 0.05, so we
  reject \(H_0\) and claim that there is significant difference between
  \texttt{merke} at 95\% confidence level.
\item
  Estimate all parameters in the model.

  The estimated parameters in the model can be obtained from model
  summary as below:

  \begin{longtable}[c]{@{}lrrrr@{}}
  \caption{Coefficient Estimate of Linear Model: \texttt{frystemp}
  \textasciitilde{} \texttt{merke}}\tabularnewline
  \toprule
  term & estimate & std.error & statistic & p.value\tabularnewline
  \midrule
  \endfirsthead
  \toprule
  term & estimate & std.error & statistic & p.value\tabularnewline
  \midrule
  \endhead
  \(\hat{\mu}\) & -14.767 & 0.234 & -63.202 & 0.000\tabularnewline
  \(\hat{\tau_1}\) & -1.396 & 0.330 & -4.224 & 0.000\tabularnewline
  \(\hat{\tau_2}\) & 1.204 & 0.330 & 3.644 & 0.002\tabularnewline
  \bottomrule
  \end{longtable}

  \begin{longtable}[c]{@{}rrrrrrr@{}}
  \caption{Summary of Linear Model: \texttt{frystemp}\textasciitilde{}
  \texttt{merke}}\tabularnewline
  \toprule
  r.squared & adj.r.squared & sigma & statistic & p.value & df &
  df.residual\tabularnewline
  \midrule
  \endfirsthead
  \toprule
  r.squared & adj.r.squared & sigma & statistic & p.value & df &
  df.residual\tabularnewline
  \midrule
  \endhead
  0.4997 & 0.4521 & 1.1446 & 10.4879 & 7e-04 & 3 & 21\tabularnewline
  \bottomrule
  \end{longtable}

  From the output, we can estimate \(\tau_3\) from the assumption that
  the overall mean sum to zero. i.e.
  \(\hat{\tau}_3 = -\left(\tau_1 + \tau_2\right) = -(-1.4+1.2) = 0.192\)
\item
  Calculate confidence intervals for \(\tau_1 - \tau_2\),
  \(\tau_1 - \tau_3\) and \(\tau_2 - \tau_3\) using Tukey's method. Are
  there any significant differences?

  The confidence intervals for Tukey's method can be computed as,

  \[
  \begin{aligned}
  \bar{y}_{i\cdot} - \bar{y}_{j\cdot} - q_\alpha (a, f)\sqrt{\frac{\text{MS}_\text{E}}{n}} &\le \mu_i - \mu_j\\
  & \le \bar{y}_{i\cdot} - \bar{y}_{j\cdot} + q_\alpha (a, f)\sqrt{\frac{\text{MS}_\text{E}}{n}} &\le \mu_i - \mu_j, i \ne j
  \end{aligned}
  \]

  since, \(\mu_i - \mu_j\) is same as \(\tau_i - \tau_j\), the
  conficence interval for group \(i\) and \(j\) can also be written as,

  \[
  \begin{aligned}
  \hat{\tau}_{i\cdot} - \hat{\tau}_{j\cdot} - q_\alpha (a, f)\sqrt{\frac{\text{MS}_\text{E}}{n}} &\le \mu_i - \mu_j\\
  & \le \hat{\tau}_{i\cdot} - \hat{\tau}_{j\cdot} + q_\alpha (a, f)\sqrt{\frac{\text{MS}_\text{E}}{n}} &\le \mu_i - \mu_j, i \ne j
  \label{eqn:tukeyCI}
  \end{aligned}
  \]

  Here we have,

  \[
  \begin{aligned}
  q_\alpha (a, f)\sqrt{\frac{\text{MS}_\text{E}}{n}} &= q_{0.05} (3, 21)\sqrt{\frac{1.310119}{8}} \\
  &= 1.4425278
  \label{eqn:tukeyCriteria}
  \end{aligned}
  \]

  Further, this value is now compated with the absolute difference
  between the estimates. The difference between the estimates of various
  group-pairs are,

  \begin{longtable}[c]{@{}rrr@{}}
  \toprule
  merke1 - merke2 & merke1 - merke3 & merke2 - merke3\tabularnewline
  \midrule
  \endhead
  -2.6 & -1.5875 & 1.0125\tabularnewline
  \bottomrule
  \end{longtable}

  If the absolute difference between two groups are larger than the
  value 1.443 than we reject the null hypothesis and claim that the pair
  are significantly different from each other. The R output for this
  test is,

\begin{verbatim}
                Lower  Center   Upper Std.Err t value  P(>t)    
merke1-merke2 -4.0425 -2.6000 -1.1575  0.5723  -4.543 0.0005 ***
merke1-merke3 -3.0300 -1.5875 -0.1450  0.5723  -2.774 0.0294 *  
merke2-merke3 -0.4300  1.0125  2.4550  0.5723   1.769 0.2042    
---
Signif. codes:  0 '***' 0.001 '**' 0.01 '*' 0.05 '.' 0.1 ' ' 1
\end{verbatim}

  Above output shows the difference between \texttt{merke2} and
  \texttt{merke1} whose interpretation will be same as the difference
  between \texttt{merke1} and \texttt{merke2} except the sign of
  Estimate and interval will switch. For instance, in this situation,
  the estimate will be -2.6 and the lower and upper confidence interval
  at 95\% confidence level will be -4.043 and -2.600 respectively.

  The result shows that \texttt{marke1} is significantly different than
  \texttt{marke2} and \texttt{marke3} at 95\% confidence level.
\end{enumerate}

\section{Exercise 2 (Topic discussed at lecture August
23)}\label{exercise-2-topic-discussed-at-lecture-august-23}

NA

We would like to investigate the fat concentration in milk
(\texttt{fettprosent} in the below data set). Three farms have been
randomly selected and 5 cows from each farm (\texttt{besetning}) is
recruited to the study. The data is as given below:

\begin{enumerate}
\def\labelenumi{\alph{enumi})}
\item
  Explain what is meant by a \emph{random effect} and why
  \texttt{besetning} (farm) reasonably can be modeled as a random
  effect.

  In many situations, factors have infinitely many levels and a
  researcher randomly choose some of the levels and make inference about
  the whole population. Models with such factors are called \emph{Random
  effect models}. In case of random effect model, the interest is on the
  population distribution of factor rather than some specific chosen
  levels.

  For example, An experiment where some specific drugs are to be tested
  for their efficacy. A randomly chosen drug will barely be an interest
  in any experiment. Here, the experiment is oriented in finding the
  effect of those specific drugs and thus, this is the case of
  \emph{fixed effect model}. While in another experiment where
  researcher is interested in finding if there is any differences
  between farm in Akershus area of Norway in the context of milk
  production. There can be many farms and the research is not interested
  on some specific farm but rather is interested on overal population of
  farm. So some farms are randomly choosen and are used to construct a
  model. This is the case of \emph{random effect model}.
\item
  Consider the model, \[
  \begin{aligned}
  y_{ij} &= \mu + \tau_i + \epsilon_{ij} \\
  \tau_i &\sim \text{NID}\left(0, \sigma_\tau^2\right)\\
  \epsilon_{ij} &\sim \text{N}\left(0, \sigma^2\right) \\
  i &= 1, 2, 3\; j = 1, 2, \ldots 5\; N = 15
  \end{aligned}
  \] Also, all random variables are independent

  Use the output below:

\begin{verbatim}
          Df Sum Sq F value    Pr(>F)    
besetning  2 13.545  35.213 9.521e-06 ***
Residuals 12  2.308                      
---
Signif. codes:  0 '***' 0.001 '**' 0.01 '*' 0.05 '.' 0.1 ' ' 1
\end{verbatim}

  Estimate \(\sigma_\tau^2\) and \(\sigma^2\) and the correlation for
  animals from the same farm.

  Here we can find \(\text{MS}_\text{treatment}\) and
  \(\text{MS}_\text{E}\) as,

  \[
  \begin{aligned}
  \text{MS}_\text{treatment} &= \frac{\text{SS}_\text{treatment}}{df_\text{treatment}}
  = 6.773 \\
  \text{and, }\text{MS}_\text{E} &= \frac{\text{SS}_\text{E}}{df_\text{Error}}
  = 0.192
  \end{aligned}
  \]

  Therefore, we can find the estimates of \emph{variance components}
  \(\hat{\sigma}_\tau^2\) and \(\hat{\sigma}^2\) as,

  \[
  \begin{aligned}
  \hat{\sigma}^2 &= \text{MS}_\text{E} = 0.192 \\
  \hat{\sigma}_\tau^2 &= \frac{\text{MS}_\text{treatment}-\text{MS}_\text{E}}{n} \\
  &= \frac{6.773 - 0.192}{5}
  = 1.316
  \end{aligned}
  \]

  Finally, the correlation for animals from the same farm is,

  \[
  \text{cor}(y_{ij}, y_{ik}) = \hat{\rho} = \frac{\hat{\sigma}_\tau^2}{\hat{\sigma}_\tau^2 + \hat{\sigma^2}} = \frac{1.32}{1.508} 
  = 0.872
  \]

  Thus the correlation for animals from the same farm is 87.2 percent.
\item
  Is there a significant effect of \texttt{besetning}? Formulate the
  hypothesis and do the test.

  To test the significance of the random factor \texttt{besetning}, the
  hypothesis can be written as,

  \[
  \begin{aligned}
  H_0 &: \sigma_\tau^2 = 0 \\
  H_1 &: \sigma_\tau^2 > 0
  \end{aligned}
  \]

  Here, \(\sigma_\tau^2\) is the variation between farms and the null
  hypothesis states that there is no variation between farms. Given
  ANOVA table shows that the p-value corresponding to \texttt{besetning}
  is very small and thus we reject null hypothesis and claim that there
  is significant difference between farms. As this is a random effect
  model, the inference is made on just those specific farms but
  population of farms in general.
\item
  Calculate a 99\% confidence interval for \(\sigma^2\)

  The confidence interval for \(\sigma^2\) at \(\alpha\) level of
  significance is,

  \[\frac{\text{SS}_E}{\chi^2_{\alpha/2, N-a}} \le \sigma^2 \le \frac{\text{SS}_E}{\chi^2_{1 - \alpha/2, N-a}}\]

  From Chi-square table, we can find chi-square values for
  \(\alpha = 0.01\) and \(N-a = 12\) as,

  \[
  \chi^2_{0.005, 12} = 28.3 \text{ and } 
  \chi^2_{0.995, 12} = 3.074
  \]

  Therefore, using \(\text{SS}_E = 2.308\), we can find,

  \[
  \begin{aligned}
  \frac{\text{SS}_E}{\chi^2_{\alpha/2, N-a}} &\le \sigma^2 \le \frac{\text{SS}_E}{\chi^2_{1 - \alpha/2, N-a}} \\
  \frac{2.308}{28.3} &\le \sigma^2 \le \frac{2.308}{3.074} \\
  0.082 &\le \sigma^2 \le 0.751
  \end{aligned}
  \]

  Thus at 99\% confidence level, the true error variance \((\sigma^2)\)
  lie between the interval \(\left[0.082, 0.751\right]\).
\end{enumerate}

\section{Exercise 3 (Lectures August
24)}\label{exercise-3-lectures-august-24}

\emph{You are supposed to use R or other statistical software for this
exercise}. The data is \texttt{comp3.dommere.RData}.

The purpose of the data of this exercise is to investigate if 5
different types of feeding (\texttt{fortype} in the below data set) lead
to differently tasting milk. Four judges (corresponding to
\texttt{Dommer} in the data) were asked to taste and rate on a scale
from 1 to 10, 10 being best.

\begin{longtable}[c]{@{}lrrrrr@{}}
\toprule
Dommer & Fortype1 & Fortype2 & Fortype3 & Fortype4 &
Fortype5\tabularnewline
\midrule
\endhead
Dommer1 & 6 & 6 & 3 & 4 & 3\tabularnewline
Dommer2 & 9 & 8 & 7 & 8 & 3\tabularnewline
Dommer3 & 10 & 8 & 5 & 7 & 6\tabularnewline
Dommer4 & 9 & 7 & 3 & 4 & 1\tabularnewline
\bottomrule
\end{longtable}

It is natural to let \texttt{Dommer} be regarded as a block effect.

Assume two different model, one includes the blocks (\emph{Model 1}),
and the other without blocking (\emph{Model 2}).

\begin{enumerate}
\def\labelenumi{\alph{enumi})}
\item
  Describe the models and state the standard assumptions. Would you
  prefer \emph{Model 1} or \emph{Model2}? Give reasons for your answer.

  \textbf{Model 1:} Let \(y_{ij}\) be the \texttt{poeng} of
  \(i^\text{th}\) fortype given by \(j^\text{th}\) dommer.

  \[y_{ij} = \mu + \tau_i + \gamma_j + \epsilon_{ij}, \text{ where, } \sum_{i = 1}^5 \tau_i = 0 \text{ and } \sum_{j = 0}^4 \gamma_j = 0\]

  Here, \(i = 1, \ldots, 5\) (Fortype) and \(j = 1, \ldots, 5\) (Dommer)

  \textbf{Model 2:} Let \(y_{ij}\) be the \(j^\text{th}\) replication of
  \texttt{poeng} for \(i^\text{th}\) fortype.

  \[y_{ij} = \mu + \tau_i + \epsilon_{ij}, \text{ where, } \sum_{i = 1}^a \tau_i = 0\]

  Here \(i = 1, \ldots, 5\) (Fortype) and \(j = 1, \ldots, 5\)
  (Replication) Since in this model dommer is not considered as blocking
  factor so, the points are considered as replications.

  The error terms \(\epsilon_{ij}\) in both \textbf{Model1} and
  \textbf{Model2} are independent and follows normal distribution with
  mean 0 and constant variance \(\sigma^2\). i.e.,

  \[\epsilon_{ij} \sim \text{NID}\left(0, \sigma^2\right)\]

  ANOVA output from \textbf{Model 1} and \textbf{Model 2} are as
  follows,

  \begin{longtable}[c]{@{}lrrrrr@{}}
  \caption{ANOVA for \textbf{Model 1}}\tabularnewline
  \toprule
  & Df & Sum Sq & Mean Sq & F value & Pr(\textgreater{}F)\tabularnewline
  \midrule
  \endfirsthead
  \toprule
  & Df & Sum Sq & Mean Sq & F value & Pr(\textgreater{}F)\tabularnewline
  \midrule
  \endhead
  Fortype & 4 & 70.30 & 17.5750 & 12.7818 & 0.0003\tabularnewline
  Dommer & 3 & 31.75 & 10.5833 & 7.6970 & 0.0039\tabularnewline
  Residuals & 12 & 16.50 & 1.3750 & &\tabularnewline
  \bottomrule
  \end{longtable}

  \begin{longtable}[c]{@{}lrrrrr@{}}
  \caption{ANOVA for \textbf{Model 2}}\tabularnewline
  \toprule
  & Df & Sum Sq & Mean Sq & F value & Pr(\textgreater{}F)\tabularnewline
  \midrule
  \endfirsthead
  \toprule
  & Df & Sum Sq & Mean Sq & F value & Pr(\textgreater{}F)\tabularnewline
  \midrule
  \endhead
  Fortype & 4 & 70.30 & 17.5750 & 5.4637 & 0.0064\tabularnewline
  Residuals & 15 & 48.25 & 3.2167 & &\tabularnewline
  \bottomrule
  \end{longtable}

  From the ANOVA tables above, we can see that the block factor
  \texttt{Dommer} has significant effect in Model 1 which indicates that
  the analysis will be affected if it is removed from the model. In
  Model 2 the MSE has increased that consiquently decreases F-value
  corresponding to Fortype.
\item
  Show that there is a significant effect of \texttt{Fortype} in
  \emph{Model 1}.

  \begin{longtable}[c]{@{}lrrrrr@{}}
  \caption{ANOVA table for \textbf{Model 1}}\tabularnewline
  \toprule
  & Df & Sum Sq & Mean Sq & F value & Pr(\textgreater{}F)\tabularnewline
  \midrule
  \endfirsthead
  \toprule
  & Df & Sum Sq & Mean Sq & F value & Pr(\textgreater{}F)\tabularnewline
  \midrule
  \endhead
  Fortype & 4 & 70.30 & 17.5750 & 12.7818 & 0.0003\tabularnewline
  Dommer & 3 & 31.75 & 10.5833 & 7.6970 & 0.0039\tabularnewline
  Residuals & 12 & 16.50 & 1.3750 & &\tabularnewline
  \bottomrule
  \end{longtable}

  Since, the p-value corresponding to \texttt{Fortype} is
  (\(3\times 10^{-4}\)) \textless{}\textless{} 0.05, there is
  significant effect of \texttt{Fortype}.
\item
  Which types of feeding (\texttt{Fortype}) differ significantly Use
  \emph{Model 1}?

  Since there is significant effect of \texttt{Fortype}, it is desirable
  to perform pairwise comparison between different \texttt{Fortype}.
  This can be done using Tukey's pariwise comparison. The output from
  Tukey's test is,

\begin{verbatim}
                   Lower Center  Upper Std.Err t value P(>t)
Fortype1-Fortype2 -1.393  1.250  3.893   0.829   1.508  0.58
Fortype1-Fortype3  1.357  4.000  6.643   0.829   4.824  0.00
Fortype1-Fortype4  0.107  2.750  5.393   0.829   3.317  0.04
Fortype1-Fortype5  2.607  5.250  7.893   0.829   6.332  0.00
Fortype2-Fortype3  0.107  2.750  5.393   0.829   3.317  0.04
Fortype2-Fortype4 -1.143  1.500  4.143   0.829   1.809  0.41
Fortype2-Fortype5  1.357  4.000  6.643   0.829   4.824  0.00
Fortype3-Fortype4 -3.893 -1.250  1.393   0.829  -1.508  0.58
Fortype3-Fortype5 -1.393  1.250  3.893   0.829   1.508  0.58
Fortype4-Fortype5 -0.143  2.500  5.143   0.829   3.015  0.07
\end{verbatim}

  The result shows that at 95\% confidence level, \texttt{Fortype1}is
  significantly different than \texttt{Fortype3}, \texttt{Fortype4} and
  \texttt{Fortype5} while \texttt{Fortype2} is significantly different
  from \texttt{Frotype3} and \texttt{Fortype5}. In addition at 90\%
  confidence level, \texttt{Fortype4} and \texttt{Fortype5} are also
  significantly different.
\end{enumerate}

NA

\begin{verbatim}
Construct a contrast that measures the difference between these two feeding regimes, and test if the concentrates types taste significantly better than the other.

<div class = 'ans'>

A contrast to test the average of `Fortype1` and `Fortype2` with average of `Fortype3` and `Fortype4` can be written as,

$$\text{Contrast: }(\Gamma) = \frac{1}{2}\left(\tau_1 + \tau_2\right) - \frac{1}{2}\left(\tau_3 + \tau_4\right) $$

The above hypothesis is written in terms of effects $\tau_i$ rather than $\mu_i$ since $\mu_1 + \mu_2 = \mu_3 + \mu_4$ is same as $\tau_1 + \tau_2 = \tau_3 + \tau_4$. Further, the coefficient of contrast is $c_i = (0.5, 0.5, -0.5, -0.5, 0)$. The hypothesis to test if the difference between concentrates types (`Fortype1` and `Fortype2`) is better than Rutabaga (`Foretype3` and `Fortype4`) is,

$$H_0: \Gamma = 0 \text{ vs } H_1: \Gamma > 0$$

The test statistic for this hypothesis is,

$$
t_0 = \frac{\sum_{i = 1}^ac_i\hat{\tau}_{i\cdot}}{\sqrt{\text{MS}_E \sum_{i = 1}^a\frac{c_i^2}{n_i}}} \sim t_{0.05, N-a}
$$ 
    
Using the test statistics above we can test the hypothesis. Test output obtained from R is as follows,


                                     Estimate   Std. Error    t value    Pr(>|t|)
----------------------------------  ---------  -----------  ---------  ----------
Fortype c=( 0.5 0.5 -0.5 -0.5 0 )        2.75     0.586302   4.690416   0.0005228

From the test output, we can see that the p-value is very small (smaller than 0.05, level of significance). The p-value here is for two-sided test and for one-sided test, this p-value will be half of the p-value in output which is even smaller. So, we reject the null hypothesis and conclude that the concentrations types are significantly better than the rutabaga type.

</div>
\end{verbatim}

\end{document}
